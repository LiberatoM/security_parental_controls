% !TEX root = atr.tex



\section{Introduction}\label{introduction}
Anycast is the practice of making an Internet address available in multiple discrete locations~\cite{rfc4786}.
This allows operators to offer services closer to clients and increase resilience through redundancy.
Examples include DNS resolvers and nameservers, and CDNs providing web caching.
Whilst the number of anycast addresses on the Internet are small (<1\%) % todo calculate number and cite manycast paper
it has a large share in Internet traffic~\cite{anycast_cdn}. % cite passive data analysis paper analyzing network traffic to anycast addresses

For this reason, efforts have been made to map anycast deployments on the Internet to \eg, measure Internet resilience.
The most notable are \manycasttwo~\cite{manycast2} and iGreedy~\cite{igreedy} that can differentiate between unicast and anycast addresses % cite both
, the latter of which can also enumerate and geolocate the Points of Presence (PoPs) behind anycast addresses.
iGreedy geolocates using latency measurements by probing an anycast address from multiple geographically distributed Vantage Points (VPs), similar to conventional IP unicast geolocation.
However, such latency measurements are known to be noisy due to \eg, network propagation delays.

Unicast IP geolocation often uses tools like traceroute to improve precision.
Traceroute allows for measuring the path that an Internet packet takes to reach its target.
When determining the location of a target, traceroute can be used to find nearby hops that have available location information.
It has also been used to geolocate anycast using the location of hops near PoPs, to infer the location of the PoPs themselves.
% an image explaining how this works (visualization of traceroute from different VPs to different anycast sites) could be useful
One example is verification of GDPR compliance~\cite{hunter}.
However, traceroute has limitations like tunneling % cite Matthew's paper on the wide usage of tunneling visible in traceroute
that keep hops hidden and may result in indeterminate geolocation results.

This work puts the traceroute technique to the test by performing it at scale towards 13k anycast prefixes. % cite our arxiv paper?
Our contributions are i) an extensive validation of the technique, ii) release of the measurement and analysis code, iii) public release of the dataset revealing up to N anycast PoPs for Y prefixes.

The paper is structured as followed.
First, in \S\ref{background} we briefly discuss background on anycast detection and traceroute and provide related work on using traceroute to detect anycast.
Then, we detail the methodology used in \S\ref{methodology} followed with an analysis of results in \S{results}.
Finally, we discuss the performance of traceroute to detect anycast and list further use cases in \S\ref{discussion}.
