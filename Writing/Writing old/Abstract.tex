In today's society, ensuring the safety of children online has become a priority for parents, educators, and policymakers. Parental control systems are widely used to restrict access to age-inappropriate online content, offering a tool for protecting children from potential online risks.
However, their blocking behavior across different technologies remains hidden and poorly analyzed.
In this paper, we present a comparative analysis of seven parental control solutions: three provided by routers with built-in filtering functionalities (TP-Link, Netgear, and ASUS), two DNS-based services (OpenDNS and DNS.eu), and a software tool (Norton Family).
Each system was tested by systematically visiting a large set of domains and recording which were blocked.
We then classified all domains into content categories and used this classification to evaluate the consistency and selectivity of each parental control system.
Our findings show that parental controls frequently fall short of providing adequate protection: Some systems fail to block up to 96\% of inappropriate domains, while others block appropriate content or apply inconsistent and arbitrary filtering rules.
For example, ASUS allows access to over 95\% of gambling websites, and Netgear exhibits internal inconsistencies, with one-third of the domains blocked under its "Adult" settings remaining accessible under its stricter "Child" setting.
These results provide insight into the real-world operation of popular parental control technologies and show a methodology for evaluating category-based filtering at scale.